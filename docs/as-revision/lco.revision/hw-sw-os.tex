\section{Software, Operating Systems, Languages}

\begin{questions}
\subsection{Software}
	\question Contrast application software with system software. 
	\question Define, with example where appropriate: 
	\begin{parts}
		\part translator;
		\part compiler;
		\part assembler;
		\part interpreter;
		\part library modules;
		\part utility program;
		\part virtual machine.
	\end{parts}

\subsection{Operating Systems}

	\question An operating systems is typified by several capabilities, define
	\begin{parts}
		\part resource management;
		\part I/O management;
		\part virtual memory;
		\part file management;
		\part UAC.
	\end{parts}
	\question Why might a user choose one operating system over another? 

\subsection{Programming Languages}
	\question Contrast machine code with assembly language. 
	\question Why might a programmer choose to code in assembly language? 
	\question Why might a programmer choose to code in a high level language? What might guide their choice of language? 
	\question With the context of high level langauges: 
	\begin{parts}
		\part imperative;
		\part functional;
		\part object-oriented;
		\part bytecode;
		\part events.
	\end{parts}


\end{questions}