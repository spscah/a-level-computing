\section{Internal Computer Architecture}

\begin{questions}
	\question System bus facts: 
	\begin{parts}
		\part How are the number of wires on the bus better known? What is a common value for this quantity? 
		\part What are the names for the three busses? 
		\part What direction will the data travel down these three busses? 
	\end{parts}
	\question How much address space is directly addressable by a 32-bit address bus? 
	\question Within the context of main memory, define volatile. 
	\question What does RAM stand for; explain the importance of R in this acronym. 
	\question Contrast the von Neumann and Harvard architectures.
	\question Give two reasons why I/O devices are handled by controllers rather than being connected directly to the processor. \begin{solution}Format. Speed. Monitoring new connections. \end{solution}  
	\question Explain why we would need ROM, RAM, a SSD and, potentially, a magnetic HDD within the same PC.
	\question On a games console, why might it take a minute to load a game?   
	\question Explain what is meant by the \emph{Stored Program Concept}. 
	\question Give a pro and a con for programs and data being considered as the same thing in memory. 
	\question What is meant by the clock speed? 
	\question Explain what purposes are served by the CIR, PC, MBR, MAR, SR in the fetch execute cycle. 
	\question What are the operands and the opcodes? 
	\question What are the two addressing modes? 
	\question What might cause an interrupt to be fired? 
	\question Describe the effect of an ISR on the fetch execute cycle. 
	\question Write assembly language instructions that would perform the following pseudocode, use registers $r_1$ to $r_n$ as necessary to store variables:
	\begin{algorithmic}
	\If{$A = 1$} 
		\State $B \gets 2$ 
	\Else 
		\State $A \gets A + 1$ 
	\EndIf
	\end{algorithmic}
\end{questions}