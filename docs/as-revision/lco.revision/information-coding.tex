\section{Information Coding}

\begin{questions}
	\question How many different ASCII codes are there? 
	\question What is the range of hex values for 
	\begin{parts}
		\part upper case letters?
		\part lower case letters?
		\part numbers?
	\end{parts}  
	\question Why has Unicode gained traction in preference to ASCII? 
	\question What would the parity bit be for the ASCII code for the letter 'A' to ensure that it passed odd parity? 
	\question Under majority voting what bits would be sent to communicate the ASCII for the number 7? 
	\question Define \emph{checkdigit} and outline either the ISBN-10, ISBN-13 or Kuhn algorithm. 
	\question In computer graphics, what is meant by the colour depth? 
	\question Relate an image's resolution to PPI. Consider the Jobsian 'Retinal Display' in your answer.
	\question Give 4 things that could be found in an image's metadata.  
	\question In a bitmapped image showing 24 bits of colour and a $1024 \times 768$ image size, how many bytes would be required? 
	\begin{solution}
	18,874,368 bits = 2.35MB
	\end{solution}
	\question For what purposes would a vector graphic format be used? Why?
	\question What attributes might you need to define to draw a circle? \begin{solution}centre (x,y), radius, fill, line, line weight, fill pattern\end{solution}
	\question Why might anti-aliasing be used on a bitmapped image after resizing?   
	\question Describe the MIDI format. 
	\question What do ADC and DAC stand for? 
	\question What aspects of sampling are required to estimate the file size? 
	\question With a bit depth of 16 bits, sampling at 44.1KHz, how many byters are required for a 60 second, single channel recording? \begin{solution}$44100 \times 60 \times 16 \div 8 = 5292000 = 5.29MB$\end{solution}    
	\question What does Nyquist's Theorem say about the relationship between sample rate and frequency? 
\end{questions}